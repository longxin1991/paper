\chapter{总结与展望}

本硕士论文结合PcP和PKiKP数据对地球核幔边界(CMB)结构变化进行研究,包括CMB的界面起伏和其上方的低速结构. 并通过一些实际的例子,分析了CMB结构变化对PKiKP/PcP振幅比和PKiKP-PcP走时残差观测的影响. 

首先利用IMS小口径台阵数据分析了CMB的小尺度起伏变化和超低速带对PKiKP/PcP振幅比观测的影响:(1) 通过NVAR和PDAR对同一地震事件的PcP和PKiKP记录比较,并结合之前波形模拟~\citep{Wu2014a}的结果,推断出CMB上可能存在局部的上凸,造成采样到不同区域的短周期PcP振幅和波形的变化;(2) 基于\citet{Rost2004a}对采样阿拉斯加Kenai半岛下方CMB被异常放大的PcP的观测,本研究结合PcP和PKiKP振幅比观测,也发现采样到Kenai半岛下方的区域的PKiKP和PcP的振幅比明显偏小,同时观察到振幅比和走时残差存在明显的相关性,从而进一步验证了前人研究的结果,即Kenai半岛下方存在局部CMB凹陷,放大了PcP振幅;(3) 通过对采样到之前研究中已经确认的澳大利亚太平洋西海岸一侧下方CMB的超低速带(ULVZ)结构~\citep{He2006a,He2012a,Thorne2004a}的WAR和ASAR数据进行分析,发现了该区域存在普遍为PREM理论值1--2倍的PKiKP/PcP振幅比,这恰好可以用之前研究得出的15\%左右的S波波速降低来解释,结合走时残差的分析,得出该CMB区域并不存在明显的界面起伏变化,从而进一步检验了CMB低速结构对PKiKP/PcP振幅比观测的影响. 

除了小口径台阵数据,本研究还利用国家测震台网观测到的PKiKP和PcP数据对位于东亚的中国下方的CMB结构变化进行分析. 通过比较筛选出的高质量PKiKP和PcP数据的振幅比和走时残差,发现与Kenai半岛的例子相似,二者存在明显的正相关性,即高的振幅比对应小的负走时残差,而低振幅比则相反. 在排除其他可能的影响因素后,本研究认为中国下方的CMB存在尺度达到200至400km的界面隆升或者凹陷,而且在大范围的隆升的界面上可能还耦合了小的凹陷,造成了PKiKP/PcP振幅比对数残差显示出负异常被正异常包围的现象. 在此观测的基础上,本文还通过对重复地震的PKiKP-PcP振幅比和走时残差观测进一步评估了观测的可靠性,并对单台站振幅比观测不确定性的主要来源进行了讨论. 

综合本研究的结果,CMB结构的各种变化均会对反射的PcP造成显著的影响,从而造成PKiKP-PcP振幅比和走时观测的区域性离散. 因此用这PcP和PKiKP研究内核外核边界结构仍然需要保持谨慎,否则将很难得到关于ICB的真实信息. 即使考虑到CMB的复杂性,单纯利用振幅比和走时残差的方法也不容易将CMB各种效应定量的分离出来. 若要得到对ICB或者CMB结构更强的约束,则需要挖掘波形中更多的信息. 目前已经有一些尝试,比如利用PKiKP和PcP谱比的方法~\citep{Tanaka2015a},利用更先进的波形模拟技术或者全波形反演方法~\citep{Fichtner2009}也可能是今后CMB或ICB结构研究的发展方向. 另外,将地震学反演和地球动力学模拟结合起来也有助于CMB复杂结构形成机制的阐明~\citep{Steinberger2008a,Soldati2012a}.