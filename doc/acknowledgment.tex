\chapter*{致谢}

有很多话要说,但摄于文字力的持久,我努力克制自己的妄言. 三年的硕士研究生学习即将结束,而我有太多需
要感谢的人. 

首先,我要感谢艾印双老师对我在学术上的指导以及生活上无微不至的关怀. 艾老师尽可能地为我们创造良好的学
习环境,勉励我不断上进. 艾老师虽然研究任务繁忙,同时身负课题组众多事务,但还是经常抽出时间和我探讨问题,给了我很多启发. 在生活上,艾老师也时时教导我做人做事的道理,让我备受感动. 如果没有艾老师的悉心指导,我是难以完成硕士论文的. 

我要感谢课题组的老师们. 郑天愉老师、何玉梅老师、赵亮老师、陈棋福老师、姜明明老师和吕彦老师,他们作为我硕士论文开题和中期报告考核小组的成员,给我的工作提了很好的意见,他们的教导都对我产生了很大的帮助. 其中,郑老师求实的治学风格给我留下了深刻的印象,她对我的鼓励是我前进的巨大动力;还有陈凌老师,每次请她帮忙,她总是不厌其烦. 虽然陈老师工作繁忙,平时我和她的交流不多,但她对我的教诲当永远不会忘记. 

我要感谢教育处的宋玉环老师、李铁胜老师和黄莹老师. 他们深刻了解学生们的学习和科研压力,经常组织我们参加各种户外活动,让我们放松身心,从而更好地投身于科研工作. 这里要特别感谢宋玉环老师,为我在去年申请到日本交流的过程中提供了很大帮助,没有她的帮忙,我必定是不能顺利成行的. 

我要感谢课题组的同学们:王旭、王武、陈瑛、魏晓拙、张耀阳师兄、徐小兵师兄、凌媛师姐还有申中寅师兄,大家
共同营造了一个团结和谐的学习环境. 其中最需要感谢的是申中寅师兄对我在学习上的帮助和研究上的鼓励. 正是申
师兄的鼓励给在处于困境中的我带来了希望,让我克服了研究工作中的重重困难,他就像黑暗中的灯塔,照亮我前进
的方向. 不仅如此,申师兄的睿智也时刻给我带来无限的启迪,这也让我对他充满了敬仰之情. 另外,我还要特别感
谢魏晓拙师弟. 他活泼开朗,谈笑风声,与他的交流总是让我感到非常愉快,让我在枯燥的科研工作中感到了不少欢
乐,支持着我度过艰难的日子. 

我还要感谢钮凤林老师,感谢他百忙之中抽出时间来听我的工作汇报. 虽然是几句不经意间的建议,却给我的研究工
作带来了很大的帮助. 通过和他的交流,可以看出他总能站在高处看待科学问题,不愧是有大家风范. 除此之外,我
还要感谢东京大学地震研究所的川勝教授,虽然只和他在去年有短短十几天的交流,但他对待科学严谨的态度给我留
下了深刻的印象,在对问题的分析方法上给了我很大启发,也正是在那之后,我的研究工作有了新的进展. 今后的研究之路上还要希望他继续指导. 

本硕士论文中理论PKiKP/PcP振幅比计算用到了Jaromir Jansky的动态射线追踪程序zrayamp~(\url{http://www.spice-rtn.org/library/software/Raytheory.html});数据处理工作用到了SAC~\citep{Goldstein2003a}和ObsPy~\citep{Beyreuther2010a},它们为本研究工作带来了极大的方便;本论文中涉及到的走时计算使用到Taup~\citep{Crotwell1999a};作图使用到了GMT~\citep{Wessel1991a}和Matplotlib~\citep{Hunter2007a},在此感谢这些软件的开发者们. 

最后,还要感谢IRIS数据中心和国家测震台网数据备份中心为本硕士论文研究提供数据支持.